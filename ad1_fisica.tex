\documentclass[a4paper, 12pt]{article}
\usepackage[top=2.5cm, bottom=2.5cm, left=2.5cm, right=2.5cm]{geometry}
\usepackage[utf8]{inputenc}
\usepackage{amsmath, amsfonts, amssymb}
\usepackage[portuguese]{babel}
\usepackage{blkarray, bigstrut}
\usepackage{setspace}
\usepackage{graphicx}
\usepackage{amsmath}
\usepackage{tikz}
\usepackage{easybmat}
\newcommand*\circled[1]{\tikz[baseline=(char.base)]{
            \node[shape=circle,draw,inner sep=2pt] (char) {#1};}}


\begin{document}
\begin{center}
Curso de Tecnologia em Sistemas de Computação \\
Disciplina : Física para Computação \\
AD1 - Segundo Semestre de 2019 \\
Nome: Fábio de Oliveira Branco\\ 
\end{center} 


\begin{enumerate}
% Questão 1
\item \begin{enumerate}
	% letra A
	\item 
	 $\vec{D_1} = 0\hat{\imath} + 25\hat{\jmath}$ \\
	
	$\vec{D_2} = -60 \, sen\,57\,\hat{\imath} + \,\,60 \, cos\,57\,\hat{\jmath}$ \\
	
	$|\vec{D_1}| = \sqrt{(25)^2} = 25$ \\
	
	$|\vec{D_2}| = \sqrt{(-60 \, sen\,57\,\hat{\imath})^2 + ( \,\,60 \, cos\,57\,\hat{\jmath})^2}$ \\
	
	$|\vec{D_2}| = \sqrt{2532,1 1067,9}$ \\
	
	$|\vec{D_2}| = \sqrt{3600}$ \\
	
	$|\vec{D_2}| = 60$ \\
	
	Para encontrar o módulo do vetor resultante utilizamos a lei de cossenos: \\ 
	
	 Resposta: $|\vec{D}_R| = \sqrt{25^2 + 2\,(25)(60)\, cos\,57} \neq 76,54km$\\
	
	% letra B
	\item $\vec{D}_{Rx}$
\end{enumerate}

% Questão 2
\item \begin{enumerate}
	\item $\displaystyle{\lim_{x\to+\infty} \sqrt[3]{\dfrac{3x + 5}{6x - 8}}} = \displaystyle{\lim_{x\to+\infty} \sqrt[3]{\dfrac{x(3 +\frac{5}{x})}{x(6 - \frac{8}{x})}}} = 		\displaystyle{\lim_{x\to+\infty} \sqrt[3]{\dfrac{3 +\frac{5}{x}}{6 - \frac{8}{x}}}}$ \\ \\
	
	Aplicando as propriedades dos limites infinitos em $\dfrac{5}{x}$ e $\dfrac{8}{x}$\\ obtemos  $\dfrac{5}{x} = 0$ \,\,\, e \,\,\, $\dfrac{8}{x} = 0$ \\ 

	Então:\\

	$\displaystyle{\lim_{x\to+\infty} \sqrt[3]{\dfrac{3 +\frac{5}{x}}{6 - \frac{8}{x}}}} = \displaystyle{\sqrt[3]{\dfrac{3 + 0}{6 - 0}}} = \displaystyle{\sqrt[3]{\dfrac{3}		{6}}} = \displaystyle{\sqrt[3]{\dfrac{1}{2}}}$ \\ \\
	
	\item $\displaystyle{\lim_{x\to+\infty} \dfrac{\sqrt{x^2 + 2}}{3x -6}} = \displaystyle{\lim_{x\to+\infty} \dfrac{\sqrt{x^2(1 + \frac{2}{x^2})}}{x(3 - \frac{6}{x})}} =  \displaystyle{\lim_{x\to+\infty} \dfrac{\sqrt{x^2} \,\,\,\,\,\, \sqrt{1 + \frac{2}{x^2}}}{x(3 - \frac{6}{x})}} =  \displaystyle{\lim_{x\to+\infty} \dfrac{x \,\,\, \sqrt{1 + \frac{2}{x^2}}}{x \,\,\, (3 - \frac{6}{x})}}$\\ \\

$= \displaystyle{\lim_{x\to+\infty} \dfrac{\sqrt{1 + \frac{2}{x^2}}}{3 - \frac{6}{x}}}$ \\ \\
	 
	 	
	Aplicando as propriedades dos limites infinitos em $\dfrac{2}{x^2}$ e $\dfrac{6}{x}$\\ obtemos  $\dfrac{2}{x^2} = 0$ \,\,\, e \,\,\, $\dfrac{6}{x} = 0$ \\ 

	Então:\\ \\
	
		 $= \displaystyle{\lim_{x\to+\infty} \dfrac{\sqrt{1 + \frac{2}{x^2}}}{3 - \frac{6}{x}}} = \displaystyle{\dfrac{\sqrt{1 + 0}}{3 - 0}} = \displaystyle{\dfrac{\sqrt{1}}{3}} = \displaystyle{\dfrac{1}{3}}$
	
	

	


	

\end{enumerate}
	
	 
\end{enumerate}

 
\end{document}
