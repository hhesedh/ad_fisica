\documentclass[a4paper, 12pt]{article}
\usepackage[top=2.5cm, bottom=2.5cm, left=2.5cm, right=2.5cm]{geometry}
\usepackage[utf8]{inputenc}
\usepackage{amsmath, amsfonts, amssymb}
\usepackage[portuguese]{babel}
\usepackage{blkarray, bigstrut}
\usepackage{setspace}
\usepackage{graphicx}
\usepackage{amsmath}
\usepackage{tikz}
\usepackage{easybmat}
\usepackage{upgreek}

\newcommand*\circled[1]{\tikz[baseline=(char.base)]{
            \node[shape=circle,draw,inner sep=2pt] (char) {#1};}}


\begin{document}
\begin{center}
Curso de Tecnologia em Sistemas de Computação \\
Disciplina : Física para Computação \\
AD1 - Segundo Semestre de 2019 \\
Nome: Fábio de Oliveira Branco\\ 
\end{center} 


\begin{enumerate}
% Questão 1
\item \begin{enumerate}
	% letra A
	\item 
	 $\vec{D_1} = 0\hat{\imath} + 25\hat{\jmath}$ \\
	
	$\vec{D_2} = -60 \, sen\,57\,\hat{\imath} + \,\,60 \, cos\,57\,\hat{\jmath}$ \\
	
	$|\vec{D_1}| = \sqrt{(25)^2} = 25$ \\
	
	$|\vec{D_2}| = \sqrt{(-60 \, sen\,57\,\hat{\imath})^2 + ( \,\,60 \, cos\,57\,\hat{\jmath})^2}$ \\
	
	$|\vec{D_2}| = \sqrt{2532,1 1067,9}$ \\
	
	$|\vec{D_2}| = \sqrt{3600}$ \\
	
	$|\vec{D_2}| = 60$ \\
	
	Para encontrar o módulo do vetor resultante utilizamos a lei de cossenos: \\ 
	
	 Resposta: $|\vec{D}_R| = \sqrt{25^2 + 2\,(25)(60)\, cos\,57} \cong 76,54km$\\
	
	% letra B
	\item $\vec{D}_{Rx} = 0 \hat{\imath} + (-50,3 \hat{\imath}) = -50,3 \hat{\imath}$ \\
	
	$\vec{D}_{Ry} = 25\hat{\jmath} + 32,7\hat{\jmath} = 57,7\hat{\jmath} $ \\
	
	$tg\theta = |\dfrac{\vec{D}_{Ry}}{\vec{D}_{Rx}}| = |\dfrac{57,7}{50,3}| \rightarrow \theta = arctg(\frac{57,7}{50,3}) \cong 49^{\circ}$ \\
	
A direção é igual a: $180^{\circ} - 49^{\circ} =  131^{\circ}$ \\

A direção e sentido é o mesmo do vetor deslocamento.

\end{enumerate}

% Questão 2
\item \begin{enumerate}
	\item Percebemos que entre [0;2] segundos a aceleração é constante, portante temos um movimento uniformemente variado.\\
	
	$S = S_0 + v_0t + a\,\dfrac{t^2}{2}$\\
	$4 = 0 + 0 \times2 + a\,\dfrac{(2)^2}{2}$\\
	$a = \dfrac{8}{4} =  2 m/s^2$\\
	
	Portanto a resposta é Sim.\\
	
	\item Fórmula para calcular a velocidade final: $v = v_0 + a \times t$ \\
	Pela questão sabemos que a aceleração é igual $2 m/s^2$.\\
	Pelo enumciado da questão o tempo é igual a $1s$.\\
	Substituindo os valores na fórmula obtemos: $v = 0 + 2 \times 1 = 2 m/s$ \\
	
	Portanto a resposta é Sim. \\

	\item Percebemos que entre [2; 4] segundos a velocidade é constante, portanto todos os ponto entre [2; 4] segundos possuem a mesma velocidade.\\
	
	A fórmula para calcular a velocidade no movimento uniforme é:\\ $v = \dfrac{\Delta s}{\Delta t} =  \dfrac{0 - 4}{4 - 2} = -2m/s$ \\ 
	
		Portanto a resposta é Sim. \\
	

\end{enumerate}
 \item $K_1:$ resultado da associação das molas em série do lado esquerdo do bloco: \\
 
  $\dfrac{1}{K_1} = \dfrac{1}{\frac{k}{2}} + \dfrac{1}{k} \Rightarrow \dfrac{1}{K_1} = \dfrac{2}{k} + \dfrac{1}{k} \Rightarrow \dfrac{1}{K_1} = \dfrac{3}{k} \Rightarrow K_1 = \dfrac{k}{3}$\\
  
  $K_2:$ resultado da associação das molas em paralelo do lado direito do bloco: \\
  
  $K_2 = \dfrac{k}{2} + \dfrac{k}{2} \Rightarrow K_2 = \dfrac{2k}{2} \Rightarrow K_2 = k$ \\
    
 $K_3:$ resultado da associação resultado da associação das molas em paralelo $K_2$ com a outra mola que está ao lado direito do bloco: \\
 
 $\dfrac{1}{K_3} = \dfrac{1}{k} + \dfrac{1}{\frac{k}{2}} \Rightarrow \dfrac{1}{K_3} =  \dfrac{1}{k} + \dfrac{2}{k}  \Rightarrow \dfrac{1}{K_3} =  \dfrac{3}{k} \Rightarrow K_3 = \dfrac{k}{3} $ \\
 
 $K_{eq}$: resultado da associação de $K_1$ e $K_3$: \\
 
  $K_{eq} = K_1 + K_3$ \\
  
  $K_{eq} =\dfrac{k}{3} + \dfrac{k}{3}$ \\
  
   $K_{eq} =\dfrac{k}{3} + \dfrac{2k}{3}$ \\

Agora podemos calcular o  período de oscilação do bloco de massa “m” através da fórmula:  $T = 2 \pi \sqrt{\dfrac{m}{K_{eq}}}$. \\

$ T =  2 \pi \sqrt{\dfrac{m}{\frac{2k}{3}}}$ \\

Resposta:  $ T =  2 \pi \sqrt{\dfrac{3m}{2k}}$ \\

  
 
  
  
\end{enumerate}

 
\end{document}
