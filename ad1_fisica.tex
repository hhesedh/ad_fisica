\documentclass[a4paper, 12pt]{article}
\usepackage[top=2.5cm, bottom=2.5cm, left=2.5cm, right=2.5cm]{geometry}
\usepackage[utf8]{inputenc}
\usepackage{amsmath, amsfonts, amssymb}
\usepackage[portuguese]{babel}
\usepackage{blkarray, bigstrut}
\usepackage{setspace}
\usepackage{graphicx}
\usepackage{amsmath}
\usepackage{tikz}
\usepackage{easybmat}
\newcommand*\circled[1]{\tikz[baseline=(char.base)]{
            \node[shape=circle,draw,inner sep=2pt] (char) {#1};}}


\begin{document}
\begin{center}
Curso de Tecnologia em Sistemas de Computação \\
Disciplina : Física para Computação \\
AD1 - Segundo Semestre de 2019 \\
Nome: Fábio de Oliveira Branco\\ 
\end{center} 


\begin{enumerate}
% Questão 1
\item \begin{enumerate}
	% letra A
	\item 
	 $\vec{D_1} = 0\hat{\imath} + 25\hat{\jmath}$ \\
	
	$\vec{D_2} = -60 \, sen\,57\,\hat{\imath} + \,\,60 \, cos\,57\,\hat{\jmath}$ \\
	
	$|\vec{D_1}| = \sqrt{(25)^2} = 25$ \\
	
	$|\vec{D_2}| = \sqrt{(-60 \, sen\,57\,\hat{\imath})^2 + ( \,\,60 \, cos\,57\,\hat{\jmath})^2}$ \\
	
	$|\vec{D_2}| = \sqrt{2532,1 1067,9}$ \\
	
	$|\vec{D_2}| = \sqrt{3600}$ \\
	
	$|\vec{D_2}| = 60$ \\
	
	Para encontrar o módulo do vetor resultante utilizamos a lei de cossenos: \\ 
	
	 Resposta: $|\vec{D}_R| = \sqrt{25^2 + 2\,(25)(60)\, cos\,57} \cong 76,54km$\\
	
	% letra B
	\item $\vec{D}_{Rx} = 0 \hat{\imath} + (-50,3 \hat{\imath}) = -50,3 \hat{\imath}$ \\
	
	$\vec{D}_{Ry} = 25\hat{\jmath} + 32,7\hat{\jmath} = 57,7\hat{\jmath} $ \\
	
	$tg\theta = |\dfrac{\vec{D}_{Ry}}{\vec{D}_{Rx}}| = |\dfrac{57,7}{50,3}| \rightarrow \theta = arctg(\frac{57,7}{50,3}) \cong 49^{\circ}$ \\
	
A direção é igual a: $180^{\circ} - 49^{\circ} =  131^{\circ}$
\end{enumerate}

% Questão 2
\item \begin{enumerate}
	\item Perceboms que entre [0;2] a aceleração é constante, portante temos um movimento uniformemente variado.\\
	
	$S = S_0 + v_0t + a\,\dfrac{t^2}{2}$\\
	$4 = 0 + 0 \times2 + a\,\dfrac{(2)^2}{2}$\\
	$a = \dfrac{8}{4} =  2 m/s^2$\\
	
	Portanto a resposta é Sim.
	
	\item Fórmula para calcular a velocidade final: $v = v_0 + a \times t$ \\
	Pela questão sabemos que a aceleração é igual $2 m/s^2$.\\
	Pelo enumciado da questão o tempo é igual a $1s$.\\
	Substituindo os valores na fórmula obtemos: $v = 0 + 2 \times 1 = 2 m/s$ \\
	
	Portanto a resposta é Sim.

	

\end{enumerate}
	
	 
\end{enumerate}

 
\end{document}
