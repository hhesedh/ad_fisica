\documentclass[a4paper, 12pt]{article}
\usepackage[top=2.5cm, bottom=2.5cm, left=2.5cm, right=2.5cm]{geometry}
\usepackage[utf8]{inputenc}
\usepackage{amsmath, amsfonts, amssymb}
\usepackage[portuguese]{babel}
\usepackage{blkarray, bigstrut}
\usepackage{setspace}
\usepackage{graphicx}
\usepackage{amsmath}
\usepackage{tikz}
\usepackage{easybmat}
\newcommand*\circled[1]{\tikz[baseline=(char.base)]{
            \node[shape=circle,draw,inner sep=2pt] (char) {#1};}}


\begin{document}
\begin{center}
Curso de Tecnologia em Sistemas de Computação \\
Disciplina : Probabilidade e Estatística \\
AP2 - Primeiro Semestre de 2020 \\
Nome: Fábio de Oliveira Branco\\ 
\end{center}


\begin{enumerate}
% Questão 1
\item \begin{enumerate}
% Letra a)
\item Para obter a distribuição de probabilidade normalizando a função, precisamos primeiramente integrar a função:\\
$\int_{1}^{3} f(x)dx = \int _1^3\dfrac{1}{4}\left(x-1\right)\left(3x-2\right)dx= \dfrac{1}{4}\times \int _1^3\left(x-1\right)\left(3x-2\right)dx=$\\
$\dfrac{1}{4}\left(\int _1^33x^2dx-\int _1^35xdx+\int _1^32dx\right)= \dfrac{1}{4} \times 3\left[\dfrac{x^3}{3}\right]^3_1 - 5\left[\dfrac{x^2}{2}\right]^3_1 + \left[2x\right]^3_1 =$\\
$\dfrac{1}{4}\left(26-20+4\right) = \dfrac{5}{2}$ \\ 

normalizando a função obtemos o resultado: \\
$f(x)= \dfrac{2}{5} \times \dfrac{1}{4}(x-1)(3x-2) = \dfrac{1}{10}(x-1)(3x-2)$

% Letra b)
\item A fórmula para calcular o valor médio da distribuição é:\\ $$\mu =  \int _{-\infty}^{\infty}x \,\, f(x) \,\, dx$$\\ 
Calculando o valor médio da distribuição através da fórmula obtemos: \\
$\mu = \int _1^3\dfrac{1}{10}x\left(x-1\right)\left(3x-2\right)dx = \dfrac{1}{10}\times \int _1^3x\left(x-1\right)\left(3x-2\right)dx =$\\
$\dfrac{1}{10}\times \int _1^33x^3-5x^2+2xdx =\dfrac{1}{10} \times 3\left[\dfrac{x^3}{3}\right]^3_1 -5\left[\dfrac{x^3}{3}\right]^3_1 + 2\left[\dfrac{x^2}{2}\right]^3_1 =  $ \\
$\dfrac{1}{10}\left(60-\dfrac{130}{3}+8\right) = \dfrac{37}{15} \cong 2,47$
\end{enumerate}
	
	 
\end{enumerate}

 
\end{document}
