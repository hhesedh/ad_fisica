\documentclass[a4paper, 12pt]{article}
\usepackage[top=2.5cm, bottom=2.5cm, left=2.5cm, right=2.5cm]{geometry}
\usepackage[utf8]{inputenc}
\usepackage{amsmath, amsfonts, amssymb}
\usepackage[portuguese]{babel}
\usepackage{blkarray, bigstrut}
\usepackage{setspace}
\usepackage{graphicx}
\usepackage{amsmath}
\usepackage{tikz}
\usepackage{easybmat}
\newcommand*\circled[1]{\tikz[baseline=(char.base)]{
            \node[shape=circle,draw,inner sep=2pt] (char) {#1};}}


\begin{document}
\begin{center}
Curso de Tecnologia em Sistemas de Computação \\
Disciplina : Probabilidade e Estatística \\
AP2 - Primeiro Semestre de 2020 \\
Nome: Fábio de Oliveira Branco\\ 
\end{center}


\begin{enumerate}
% Questão 1
\item \begin{enumerate}
% Letra a)
\item Para obter a distribuição de probabilidade normalizando a função, precisamos primeiramente integrar a função:\\
$\int_{1}^{3} f(x)dx = \int _1^3\dfrac{1}{4}\left(x-1\right)\left(3x-2\right)dx= \dfrac{1}{4}\times \int _1^3\left(x-1\right)\left(3x-2\right)dx=$\\
$\dfrac{1}{4}\left(\int _1^33x^2dx-\int _1^35xdx+\int _1^32dx\right)= \dfrac{1}{4} \times 3\left[\dfrac{x^3}{3}\right]^3_1 - 5\left[\dfrac{x^2}{2}\right]^3_1 + \left[2x\right]^3_1 =$\\
$\dfrac{1}{4}\left(26-20+4\right) = \dfrac{5}{2}$ \\ 

normalizando a função obtemos o resultado: \\
$f(x)= \dfrac{2}{5} \times \dfrac{1}{4}(x-1)(3x-2) = \dfrac{1}{10}(x-1)(3x-2)$

% Letra b)
\item A fórmula para calcular o valor médio da distribuição é:\\ $$\mu =  \int _{-\infty}^{\infty}x \,\, f(x) \,\, dx$$\\ 
Calculando o valor médio da distribuição através da fórmula obtemos: \\
$\mu = \int _1^3\dfrac{1}{10}x\left(x-1\right)\left(3x-2\right)dx = \dfrac{1}{10}\times \int _1^3x\left(x-1\right)\left(3x-2\right)dx =$\\
$\dfrac{1}{10}\times \int _1^33x^3-5x^2+2xdx =\dfrac{1}{10} \times 3\left[\dfrac{x^3}{3}\right]^3_1 -5\left[\dfrac{x^3}{3}\right]^3_1 + 2\left[\dfrac{x^2}{2}\right]^3_1 =  $ \\
$\dfrac{1}{10}\left(60-\dfrac{130}{3}+8\right) = \dfrac{37}{15} \cong 2,47$

% Letra c)
\item Para calcular a variãncia da distribuição iremos usar a seguinte fórmula: \\
$$\sigma^2 = \int_{-\infty}^{\infty} x^2 \,\, f(x) \,\, dx \,\, - \mu^2$$\\

Calculando o resultado: \\
$\int _1^3\dfrac{1}{10}x^2\left(x-1\right)\left(3x-2\right)dx = \dfrac{1}{10}\times \int _1^3x^2\left(x-1\right)\left(3x-2\right)dx$\\
$\dfrac{1}{10}\times \int _1^33x^4-5x^3+2x^2dx = \dfrac{1}{10}(3\left[\dfrac{x^5}{5}\right]^3_1 - 5\left[\dfrac{x^4}{4}\right]^3_1 + 2\left[\frac{x^3}{3}\right]^3_1) = $ \\
$\dfrac{1}{10}(\dfrac{726}{5} - 100 + \dfrac{52}{3}) = \dfrac{469}{75}$\\ \\
Agora podemos calcular a variância: \\ 
$\sigma^2 = \dfrac{469}{75} -2,47^2 \cong 0,15$ 
\end{enumerate}
%Fim questão 1 
\newpage
% Questão 2
\item \begin{enumerate}
% letra a
\item Para verificar se as expressões são distriuições de probabilidade temos que integrar a função, caso o resultado seja 1  então podemos comprovar que a expressão é uma distribuição de probabilidade. \\

$\int _{-1}^06\left(x^2-1\right)\left(x-x^3\right)dx = 6 \times \int _{-1}^0\left(x^2-1\right)\left(x-x^3\right)dx =$\\
$6\times \int _{-1}^0-x^5+2x^3-xdx = 6\left(-\int _{-1}^0x^5dx+\int _{-1}^02x^3dx-\int _{-1}^0xdx\right) = $\\
$6\left(-\left(-\frac{1}{6}\right)-\frac{1}{2}-\left(-\frac{1}{2}\right)\right) =  1$\\ \\
Verificamos que esta expressão é uma distribuição de probabilidade.\\ 
%letra b
\item $\int _2^4\left(x-3\right)\left(x-2\right)+1dx = \int _2^4x^2-5x+7dx = \int _2^4x^2dx-\int _2^45xdx+\int _2^47dx = $ \\
$\dfrac{56}{3}-30+14 = \dfrac{8}{3}$ \\ \\

Normalizando a função para obter a distribuição de probabilidade: \\
$f(x) = \dfrac{3}{8}(x-3)(x-2)+1$ \\

% letra c
\item $\int _0^1e^x-edx = \int _0^1e^xdx-\int _0^1edx = e-1-e = -1$ \\ \\
Através da integração da função acima percebemos que a expressão não é uma função de probabilidade, pois seu resultado é menor que zero.

\end{enumerate}	
% Questão 3
\item \begin{enumerate}
% letra a)
\item $P(< 39) = P(Z < - 0,095) = 0,5 - 0,0359 = 0,4651$ \\
Resultado aproximado $46,41 \%$ \\

\item $P(x > 42) = P(Z > \dfrac{42-41}{21}) = P(Z > 0,0476) = 0,5 - 0,0119 = 0,4801$
Resultado aproximado $48,01 \%$ \\

\item $C, P(40 < x < 43) = P(\dfrac{40 - 41}{21} < Z < \dfrac{43 - 41}{21}$ \\
$P(-0,0476 < Z < 0,095) = 0,0199 + 0,0359 = 0,0558$ \\
Resultado aproximado $5,58 \%$ \\


\end{enumerate}
	 
\end{enumerate}

 
\end{document}
